\documentclass[10pt,a4paper]{moderncv}

\moderncvstyle{classic}
\moderncvcolor{orange}

\usepackage[utf8]{inputenc}
\usepackage[T1]{fontenc}
\usepackage[inline]{enumitem}
\usepackage[scale=0.8,top=0.5cm, bottom=0.5cm, left=1.5cm, right=1.5cm]{geometry}

\setlength{\hintscolumnwidth}{3cm}

\firstname{Maxime}
\familyname{Ben Hassen}
\title {Développeur full-stack}
\address{Genève}{France}
\email{maxime.benhassen@fedabian.fr}
\photo[45pt]{photo-cv.jpg}
\homepage{fedabian.fr}



\begin{document}

  \makecvtitle
  \vspace*{-3em}
  \section{Formation}
  \cventry[0.5em]{2023-2025}{\href{https://www.informatique.univ-paris-diderot.fr/formations/masters/genial/accueil}{Master GÉNIe logiciel en ALternance (GÉNIAL)}}{}{}{}{Université Paris Cité (Paris 13ème, France) en alternance à la \href{https://sgp.fr}{Société des grands projets}}
  \cventry[0.5em]{2022-2023}{Troisième année de licence}{}{}{}{Université Laval (Québec, Canada)}
  \cventry[0.5em]{2021-2023}{Licence informatique et applications}{mention bien}{}{}{Université Paris Cité (Paris 6ème, France)}
  \cventry[0.5em]{2020-2021}{Première année de licence mathématiques-informatique}{mention très bien}{}{}{Université Claude Bernard, Lyon 1 (Villeurbanne, France)}
  \cventry[0.5em]{2017-2020}{Baccalauréat scientifique, spécialité mathématiques}{mention très bien}{}{}{Lycée Aristide Briand (Gap, France)}


  \section{Expérience professionnelle}
  \cventry[0.5em]{2023 - 2025}{Chargé de développement informatique}{}{\href{https://sgp.fr}{Société des grands projets} (St-Denis, France)}{}{
  \begin{itemize}
    \setlength\itemsep{-0.5em}
    \item Réalisation de développements (PHP, Symfony, Pimcore, Java, Springboot, VueJS)
    \item Mise en place de CI/CD (Docker, Gitlab)
  \end{itemize}
  }
  \cventry[0.5em]{2022 - 2023}{Développeur freelance}{}{}{}{
  \begin{itemize}
    \setlength\itemsep{-0.5em} 
    \item \href{https://ergo-art-senaux.ch}{Les Arsenaux Sàrl (ergo-art-senaux.ch)} (Wordpress, PHP, TypeScript, Sass)
    \item Devichy Avocats (devichyavocats.com) (Wordpress, PHP, TypeScript, Sass)
  \end{itemize}
  }
  \cventry[0.5em]{2021 - 2022}{Développeur chez B.I.O.S (Junior entreprise de l’Université Paris-Cité)}{}{}{}{
  \begin{itemize}
    \setlength\itemsep{-0.5em}
    \item Consultant pour la junior entreprise sur diverses missions
    \item Programmation et développement (PHP, Symfony, React)
  \end{itemize}
  }
  \cventry[0.5em]{2020 - 2021}{Mon paysan alpin, magasin de produits locaux}{}{}{}{
  \begin{itemize}
    \setlength\itemsep{-0.5em}
    \item Vente de produits
    \item Entretien du magasin
  \end{itemize}}

  \section{Expérience associative}
  \cventry[0.5em]{2021 - 2025}{TokTok - Application Web (\href{https://toktok.actionpopulaire.fr}{toktok.actionpopulaire.fr})}{}{}{}{
    \begin{itemize}
      \setlength\itemsep{-0.5em}
      \item Développeur en chef pour le \href{https://discord-gauche.fr/}{Discord Insoumis} (React, TypeScript, Python, Django, Docker, GitlLab-ci)
      \item Réflexion autour de l’évolution du projet et des fonctionnalités à ajouter
    \end{itemize}
  }
  \cventry[0.5em]{2020 - 2025}{Collectif Contribulle (\href{https://contribulle.org}{contribulle.org})}{}{}{}{
    \begin{itemize}
      \setlength\itemsep{-0.5em}
      \item Développeur et intégrateur frontend (HTML, SCSS, JS, PHP, Laravel)
      \item Réflexion autour de l’évolution de projet, du design et des fonctionnalités à ajouter
    \end{itemize}}
  \cventry[0.5em]{2019 - 2022}{Bénévole lors d’\emph{install party}, installation et dépannage de distributions Linux}{}{}{}{
    \begin{itemize}
      \setlength\itemsep{-0.5em}
      \item Au sein de l’association Ubuntu-fr (Paris, France) 
      \item Lors des Journées Du Logiciel Libre (Lyon, France)
    \end{itemize}
  }
  \cventry[0.5em]{2018 - 2021}{Le pas de l’oiseau, compagnie de théâtre professionnelle}{}{}{}{
    \begin{itemize}
      \setlength\itemsep{-0.5em}
      \item Bénévole, montage et démontage de plateaux
    \end{itemize}}
  \cventry[0.5em]{2016 - 2024}{Relais court-circuit La Piarre (\href{https://beta.cclapiarre.fedabian.fr}{cclapiarre.fedabian.fr})}{}{}{}{
    \begin{itemize}
      \setlength\itemsep{-0.5em}
      \item Développeur fullstack (Javascript, VueJs, Python, Django)
    \end{itemize}    
  }

  \section{Compétence personnelle}
  \subsection{Langues}
  \cvline{Français}{Langue maternelle}
  \cvline{Anglais}{Niveau B2}

  \subsection{Compétence informatique}
  \cvcomputer{Langages}{
    Python : Notions avancées\newline
    C : Notions avancées\newline
    Java : Notions avancées\newline
    Golang : Notions basiques
  }{Web}{HTML5 - CSS3 (SCSS) - JavaScript/Typescript (VueJs, NuxtJs, React) - Golang - PHP (Laravel, Symfony) - Python (Django, Flask)}
  \cvcomputer{OS}{GNU/Linux : Arch-Linux, Fedora (Silverblue), Debian}{Outils}{Docker (Podman), GitLab-ci, Github-actions, Latex, VsCode, JetBrains}
  \cvcomputer{Base de données}{MariaDB, PostgreSQL, MSSQL, Redis}{}{}

  \subsection{Certification}
  \cvline{2021}{API Society « Programmation Python - les fondamentaux »}

\end{document}
