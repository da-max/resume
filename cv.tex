\documentclass[10pt,a4paper]{moderncv}

\moderncvstyle{classic}
\moderncvcolor{orange}

\usepackage[utf8]{inputenc}
\usepackage[T1]{fontenc}
\usepackage[inline]{enumitem}
\usepackage[scale=0.8,top=0.5cm, bottom=0.5cm, left=1.5cm, right=1.5cm]{geometry}

\setlength{\hintscolumnwidth}{3cm}

\firstname{Maxime}
\familyname{Ben Hassen}
\title {Freelance en informatique}
\address{34 avenue Anatole France}{94600 Choisy-le-roi}{France}
\email{maxime.benhassen@tutanota.com}
\photo[64pt]{photo-cv.jpg}
\mobile{06 59 61 71 69}
\homepage{fedabian.fr}

\begin{document}

  \makecvtitle
  \vspace*{-1em}
  \section{Formations univesitaires}
  \cventry[1em]{2021-2022}{Deuxième année de licence informatique et applications}{en cours}{}{}{Université Paris Cité (Paris 6ème, 75)}
  \cventry[1em]{2020-2021}{Première année de licence mathématiques-informatique}{mention très bien}{}{}{Université Claude Bernard, Lyon 1 (Villeurbanne, 69)}
  \cventry[1em]{2017-2020}{Baccalauréat scientifique, spécialité mathématiques}{mention très bien}{}{}{Lycée Aristide Briand (Gap, 05)}


  \section{Expériences professionnelles}
  \cventry[1em]{2021 - 2022}{Développeur chez \href{https://bios-paris.fr/}{B.I.O.S}}{}{}{}{
  \begin{itemize}
    \setlength\itemsep{-0.2em}
    \item Consultant pour la junior entreprise sur diverses missions
    \item Programmation et développement
  \end{itemize}
  }
  \cventry[1em]{2020 - 2021}{Mon paysan alpin, magasin de produits locaux}{}{}{}{
  \begin{itemize}
    \setlength\itemsep{-0.2em}
    \item Vente de produits
    \item Entretien du magasin
  \end{itemize}}
  % \cventry[1em]{2016}{Théâtre « La Passerelle » (Gap, 05)}{}{}{}{
  %   \begin{itemize}
  %     \setlength\itemsep{-0.2em}  
  %     \item Découverte du métier de régisseur son/lumière/plateau
  %   \end{itemize}
  % }

  \section{Expériences associatives}
  \cventry[1em]{2021 - 2022}{\href{https://toktok.actionpopulaire.fr}{TokTok} - Application Web}{}{}{}{
    \begin{itemize}
      \setlength\itemsep{-0.2em}
      \item Développeur frontend pour le \href{https://discord-insoumis.fr/}{Discord Insoumis} (HTML, CSS, TypeScript)
      \item Réflexion autour de l’évolution du projet et des fonctionnalités à ajouter
    \end{itemize}
  }
  \cventry[1em]{2020 - 2022}{Collectif Contribulle}{}{}{}{
  \begin{itemize}
    \setlength\itemsep{-0.2em}
    \item Développeur et intégrateur frontend pour le site \href{https://contribulle.org}{contribulle.org} (HTML, SCSS, JS)
    \item Refléxion autour de l’évolution de projet, du design et des fonctionnalités à ajouter
  \end{itemize}}
  \cventry[1em]{2019 - 2022}{Bénévole lors d’\emph{install party}, installation et dépannage de distributions linux}{}{}{}{
    \begin{itemize}
      \setlength\itemsep{-0.2em}
      \item Au sein de l’association Ubuntu-fr (Paris, 75) 
      \item Lors des Journées Du Logiciel Libre (Lyon, 69)
    \end{itemize}
  }
  \cventry[1em]{2018 - 2021}{Le pas de l’oiseau, compagnie de théâtre professionnelle}{}{}{}{
    \begin{itemize}
      \setlength\itemsep{-0.2em}
      \item Bénévole, montage et démontage de plateaux
      \item Membre du conseil d’administration
    \end{itemize}}
  \cventry[1em]{2016 - 2021}{Relais court-circuit La Piarre, association de vente de produits locaux}{}{}{}{
    \begin{itemize}
      \setlength\itemsep{-0.2em}
      \item Développeur fullstack du site \href{https://cclapiarre.deblan.fr}{cclapiarre.deblan.fr} (VueJs, Django)
    \end{itemize}    
  }

  \section{Compétences personnelles}
  \subsection{Langues}
  \cvline{Français}{Langue maternelle}
  \cvline{Anglais}{Niveau B1}
  \cvline{Italien}{Niveau A2}

  \subsection{Compétences informatiques}
  \cvcomputer{Langages}{
    Python : Notions avancées\newline
    C : Notions avancées\newline
    Java : Notions avancées\newline
    Golang : Notions basiques
  }{Web}{HTML5 - CSS3 (SCSS) - JavaScript/Typescript (VueJs, NuxtJs) - Golang et Python (Django, Flask)}
  \cvcomputer{OS}{GNU/Linux : Arch-Linux, Fedora (Silverblue), Debian}{Outils}{Docker (Podman), GitLab, Latex, LibreOffice, VsCode, PyCharm}
  \cvcomputer{Base de données}{MariaDB, PostgreSQL, MongoDB}{}{}

  \subsection{Certifications}
  \cvline{2021}{API Society « Programmation Python - les fondamentaux »}

\end{document}
